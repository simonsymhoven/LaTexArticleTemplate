\newcommand{\DeinName}{Vorname Nachname}
\newcommand{\TitelArbeit}{Titel der Arbeit}
\newcommand{\Labor}{Seminar/Labor/Modul DevOps}
\newcommand{\PrueferEins}{Prof. Dr. Erster Pr�fer}
\newcommand{\PrueferZwei}{Dipl.-Ing Zweiter Pr�fer}
\newcommand{\fachbereich}{Informatik und Mathematik}
\newcommand{\Datum}{\today}

\makeglossary


%
% Einbinden des Headers, hier k�nnen auch weitere Einstellungen vorgenommen werden.
\input{source/header}
% Beginn des Dokuments
\begin{document}
\nocite{*}

% Einbinden des Deckblatts, der Eidesstattlichen Erkl�rung und dem Abstract
\frontmatter
% erste Seite (Titelseite) der Diplomarbeit mit Titel, Name usw...

\newsavebox{\Prof}
\savebox{\Prof}{Erster Pr�fer }

\newsavebox{\Betr}
\savebox{\Betr}{Zweiter Pr�fer }



\begin{titlepage}
%\begin{textblock}{50}(150,30)
  %\includegraphics[height=3cm]{source/images/BO-Logo_m_Wortmarke_P10cmWebHQ}
	%\includegraphics[height=1cm]{source/images/BO-Logo_m_Wortmarke_L10cmWebHQ}	
%	\includegraphics[height=4cm]{source/images/BO-Logo_m_Wortmarke_P10cmWebHQ}
%\end{textblock}
\begin{center}

{\large Hochschule M�nchen}\\[1mm]
{\small University of Applied Sciences}\\
{\small Fachbereich \fachbereich}
\vspace{1.5cm}

{\huge \TitelArbeit}

\vspace{2cm}

{\large Bachelorarbeit / Masterarbeit}

\vspace{1.5cm}

von\\[2mm]

\textbf{\large{\DeinName}}\\
\vspace{1.5cm}
\Labor\\[1cm]

Betreuer: \\[2mm]

\usebox{\Prof \PrueferEins} \\
\usebox{\Betr \PrueferZwei} \\

\vspace{1.5cm}
M�nchen, den \today
\end{center}

\begin{textblock}{50}(20,250)
	\includegraphics[height=3cm]{source/images/hmlogo}	
\end{textblock}

% Zweites Bild
%% Hier kann nat�rlich auch jedes andere Logo eingebunden werden! %%
% \begin{textblock}{50}(100,250)
  %\includegraphics[height=3cm]{source/images/hmlogo.png}
%\end{textblock}

\end{titlepage}
\chapter{Eidesstattliche Erkl\"arung}
Hiermit erkl\"are ich, \Author, die vorliegende Modularbeit selbstst\"andig und nur unter Verwendung der von mir angegebenen Literatur verfasst zu haben. 

Die aus fremden Quellen direkt oder indirekt \"ubernommenen Gedanken sind als solche kenntlich gemacht.

Diese Arbeit hat in gleicher oder \"ahnlicher Form keiner anderen Pr\"ufungsbeh\"orde vorgelegen.\\
\\[6ex]

M\"unchen, den \today


\rule[-0.2cm]{5cm}{0.5pt}

\textsc{\Author} 


\chapter{Abstract}
\label{Abstract}
Der Abstract sollte auf Englisch verfasst werden. Er ist eine komprimierte Wiedergabe der wesentlichen Erkentnisse die w�hrend der Abschlussarbeitsphase gewonnen wurden. Der Abstract soll dem Leser eine Entscheidungsgrundlage liefern, ob der Text f�r ihn/sie interessant und lesenswert ist.

% Einbinden des Inhaltsverzeichnisses
%\todo[inline]{Inhaltsverzeichnis-Schachtelungswerte auf Standard zur�cksetzen!}
%\setcounter{secnumdepth}{4} % Schachtelungstiefe der Nummerierung von �berschriften
%\setcounter{tocdepth}{4} % Schachtelungstiefe des Inhaltsverzeichnisses
\tableofcontents 

%Einbinden des Hauptteils
\mainmatter
\chapter{Einleitung}
\section{Motivation}
Hier sollten die folgenden Fragestellungen bearbeitet werden:

\begin{itemize}
	\item Warum soll das Thema bearbeitet werden, was war der ausschlaggebende Grund dazu?
	\item Gibt es vll. geleistete Vorarbeit aus Labortätigkeit, oder wurde das Thema von einem Betreuuer vorgeschlagen?
	\item In welchem Kontext steht das Thema zur akademische Ausbildung?
	\item Warum ist es sinnvoll das Thema zu bearbeiten?
\end{itemize}

\section{Aufgabenstellung}
Wie lautet die Aufgabenstellung? \cite{Nobody06}
\chapter{Grundlagen}
\section{Erster Abschnitt}
\section{Zweiter Abschnitt}

%Das Fazit
\chapter{Fazit}
\section{Zusammenfassung}
Was wurde im Rahmen umgesetzt? Anteil der eigenen Entwicklung, Abgrenzung gegenüber Fremdleistungen

\section{Ausblick}
Wie sieht die Weiterentwicklung aus? Welche Möglichkeiten der Erweiterung / Verbesserung bietet das neue Verfahren / die Software?

%Einbinden des Abbildungsverzeichnisses
\backmatter
%Liste der Tabellen
\listoftables
%Einbinden des Tabellenverzeichnisses
\listoffigures
%Einbinden des Sourcecodeverzeichnisses
\lstlistoflistings

% Quellenverzeichnis
\phantomsection
\addcontentsline{toc}{chapter}{Literaturverzeichnis}
\appendix
\bibliographystyle{plain}
\bibliography{source/bib/references.bib}

% Anhang
\appendix
\chapter{Danksagung}
Dieser Text bietet sich an für eine Danksagung. Bei Bachelorarbeiten ggf. auskommentieren... 

\end{document}
